\label{future}
\section{Future possibilities}
Looking at the 35 bugs that are targeted at version 1.0 reveals the future possibilities of the project. Many of the bugs that are filed are the result of comments that were made by current users of the system. A comment that several users have made is that using the more advanced Boolean connectors is too complicated, and that a query builder would be a useful tool. Another suggestion that users have made is for the corpus to expand horizontally, such that it encompasses more courts. Monitoring the site has also revealed that many users are not creating accounts on the site, and thus are using it primarily as a search tool. An open question is how to convert these visitors into registered users more consistently. 

Another question that remains open is the long-term costs of the project. It is currently a relatively inexpensive operation, but if it becomes popular, it could become very expensive very quickly, and due to its daily aggregation of additional content, it will soon need more hardware to hold the database, PDFs and indexes. Some monetization ideas have been drawn up for the site, and can be implemented if enough users begin using it. These ideas range from advertising on the site to premium services for extreme users. Keeping the site free is a priority, so implementing these ideas carefully is a must.

During the next few months I will be analyzing these options for future development, and will be selecting those options that provide the most value to the system. As the site grows in popularity and features, it will be necessary to recruit additional developers to expand and maintain the features of the site, but at its current state it provides a much-needed tool to the legal research community, filling a gap that was inadequately served by most other systems, and costly when done well.
