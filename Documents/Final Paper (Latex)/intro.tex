\section{Introduction}
Since the late 18th century, legal Legal documents have been in the American public record for centuries, and there is widespread belief that the legal system functions best when access to these documents is efficient and accurate. 
Although for the first century of American history there were In 1880, the West Publishing group began to address this need by creating the first volume of the Federal Register. In it, they placed the most recent opinions from the court, and in their introduction, they explain their purpose:
\begin{quote}
\end{quote}
Recent innovations in computer technology make addressing this need possible, and allow legal corpora to be rapidly and accurately searched, however existing solutions fall short of the needs of many users.
People who wish to stay up to date in certain areas of the law lack tools to do so because the existing solutions are overly complicated, too expensive, and unreliable. 
In addition, historical court opinions are often difficult to find on court websites. In this paper, I present a new system named CourtListener.com, which automatically visits court websites, downloads and categorizes the latest judgments, and then either sends alerts to users or generates custom Atom feeds. The beta version of this system has been running for more than a month, and is fully functional. It has has also been populated with more than 130,000 judgments, creating a large and growing corpus for users to query.



Although legal documents have been part of the public record in America for centuries, it is only recently that technology has facilitated the rapid and accurate search and retrieval of those documents. However, although most of these documents are available for free by visiting court websites, because of the decentralized nature of the American court system, people must visit many sites simply to stay up to date on a topic that they find relevant. For those people that are only interested in more obscure areas of the law, this may mean visiting a court website every day or week for months, simply to check if their topic has been mentioned. This means that they must read dozens or hundreds of documents that are of little or no value. Compounding this challenge is the fact that many courts do not make historical documents readily accessible, and so precedent-setting judgments can become hidden from view. Companies have attempted to address these problems by providing alerts to users and massive legal corpora, however current systems are difficult to use and understand and are behind expensive paywalls, which lay people will not pay for and many laywers, academics and non-profits cannot afford. In this paper I present a new system, CourtListener.com, which automatically visits court websites, downloads and categorizes the latest judgments, and then either sends alerts to users or generates custom Atom feeds. The beta version of this system has been running for more than a month, and is fully functional. It has has also been populated with more than 130,000 judgments, creating a large and growing corpus for users to query.





Although legal documents have been part of the public record in America for centuries, it is only recently that technology has facilitated the rapid and accurate search and retrieval of those documents. However, although most of these documents are available for free by visiting court websites, because of the decentralized nature of the American court system, people must visit many sites simply to stay up to date on a topic that they find relavant. For those people that are only interested in more obscure areas of the law, this may mean visiting a court website every day or week for months, simply to check if their topic has been mentioned. 
This means that they must read dozens or hundreds of documents that are of little or no value. Compounding this challenge is the fact that many courts do not make historical documents readily accessible, and so precedent-setting judgments can become hidden from view. 
Companies have attempted to address these problems by providing alerts to users and massive legal corpora, however current systems are difficult to use and understand and are behind expensive paywalls, which lay people will not pay for and many laywers, academics and non-profits cannot afford. In this paper I present a new system, CourtListener.com, which automatically visits court websites, downloads and categor


