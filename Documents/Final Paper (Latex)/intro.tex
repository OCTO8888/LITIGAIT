\label{intro}
\section{Introduction}
At least as far back as the early thirteenth century, there has been an awareness of the need for legal documents to be organized, cataloged, and archived. The first known English-language case reports were known as the Year Books, which were prepared in England from 1292 to 1535.\cite{berring_legal_1987} Since that time, the collection of such information has been inconsistent, and has changed format numerous times. 
In America, the cataloging of legal documents likely began with Ephraim Kirby's Report of Cases,\cite{kirby_reports_1788} in 1785, however it was not until 1880, when the West Publishing group began their Federal Reporter series that a consistent, complete and well-organized catalog of Federal cases was created. Since that time, West's Federal Reporter has become the de facto source of legal citations, and has become known as a careful and complete source of Federal cases. 

In the 1970's however, there was a revolution in the ways that lawyers and academics accessed legal documents, as Computer-Aided Legal Research (CALR) became an increasingly powerful possibility.\footnote{Although it was not until the 1980's that they became commercially viable. Originally, there was a per search cost of up to \$5,000 for queries that are trivial by today's standards.}$^{\textrm{, }}$\cite{harrington_brief_1984} While there was initially much debate in the law librarian field as to the merits of such research methods, with some expressing outright scorn for the new systems, for the most part, the debate has subsided, with most researchers and lawyers accepting the merits of the new systems.

With these new systems gaining in popularity, and with the ever-decreasing cost of computer hardware and software, a new niche has emerged for free legal research tools and corpora. There are currently a handful of such tools available on the market, including Google Scholar, Resource.org, FindLaw, Justia, LexisOne, and, until recently, AltLaw.\footnote{As of 3 May 2010, AltLaw has posted a notice in their site stating, ``AltLaw.org has shut down, permanently. We would like to thank everyone for their support.'' A cited reason in their explanation is Google's recent entry into the legal research field.}
While the systems with the largest corpora are not yet free to the public, this is nevertheless a huge development in the law, as, for the first time in history, it lowers the barriers of legal research such that lay people can easily complete much of the same research as professional legal scholars and attorneys. As a result, the legal world is opened widely to the public, aiding in their understanding of the law, and allowing them to research matters that are of interest to them.

One function that these tools lack, however, is a method for their users to stay up to date with new cases as they are issued by the courts. This leaves researchers with few options if they want or need to stay up to date with an area of the law or with a series of cases. One option that they have is to subscribe to mailing lists (electronic or otherwise), which aim to keep lawyers up to date with certain areas of the law by sending regular highlights of cases that they feel are relevant.\footnote{During one of the user interviews, this was also discussed as a method of demonstrating awareness of changes in the law, in the even of a malpractice lawsuit.} This can be a free or inexpensive approach for staying up to date, but the choice of material is not in the hands of users, and separating the wheat from the chaff can be time-consuming, at best. Another option that is available for users is to use existing alert systems, such as Google Alerts, however these can be highly unreliable, as users are subject to the tool's crawl rate, which can take a very long time to discover new content, or which can omit relevant information altogether. A final option that is available to supplement or replace the first two, is to simply visit the court websites on a regular basis, and to check there for any new content of interest. For the most part, this is approach works, though it requires a considerable amount of effort, and some courts do not freely publish all of their documents.

In this paper, I introduce a new product, CourtListener.com, which aims to ease this problem by providing a free and open source platform for the aggregation, organization, search and retrieval of legal documents. The aggregation of new court documents is completed by a daemon on a rolling basis, building a huge corpus, and providing the latest cases from the Federal Courts of Appeal within -- on average -- about fifteen minutes from the moment they are published on the court website. From there, the documents are quickly indexed, and RSS feeds and document listings are updated. Finally, at the close of each day and beginning of each week and month, alerts are emailed to registered users informing them about topics that they have identified as relevant. More details about the creation of the corpus, and the design decisions that went into this are available in section \ref{techdecisions}.

In building this system, I spoke with a number of lawyers and academics to understand their needs, and to get input into the design of the system. I will discuss the findings of these informal interviews in section 3, below. Further, after releasing the beta version of the platform, I have received some feedback from users, which I will discuss in 5, which is devoted to discussions of the future of the platform.
