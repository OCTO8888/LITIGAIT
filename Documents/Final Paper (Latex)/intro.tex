\label{intro}
\section{Introduction}
At least as far back as the early thirteenth century, there has been an awareness of the need for legal documents to be organized, cataloged, and archived. The first known English-language case reports were known as the Year Books, which were prepared in England from 1292 to 1535.\cite{berring-page-17} Since that time, the collection of such information has been inconsistent, and has changed forms a number of times. 
In America, it likely began with Ephraim Kirby's Report of Cases,[INSERT CITE AND ITALICS], in 1785, however it was not until 1880, when the West Publishing group began their Federal Reporter series that a consistent, complete and well-organized catalog of Federal cases was created. Since that time, West's Federal Reporter has become the de facto source of legal citations, and has become known as a careful and complete source of Federal cases. 

In the 1970's however, there was a revolution in the ways that lawyers and academics accessed legal documents, as Computer-Aided Legal Research (CALR) became an increasingly powerful possibility.[INSERT CITE TO HARRINGTON]\footnote{Although it was not until the 1980's that they became commercially viable. Originally, there was a per search cost of up to \$5,000 for queries that are trivial by today's standards.} While there was initially much debate in the law librarian field as to the merits of such research methods, with some expressing outright scorn for the new systems, for the most part, the debate has subsided, with most researchers and lawyers accepting the merits of the new systems.

With these new systems gaining in popularity, and with the ever-decreasing cost of computer hardware and software, a new niche has emerged for free legal research tools and corpora. There are currently a handful of such tools available on the market, including Google Scholar, Resource.org, FindLaw, Justia, LexisOne, and, until recently, AltLaw.\footnote{As of 3 May 2010, AltLaw has posted a notice in their site stating, ``AltLaw.org has shut down, permanently. We would like to thank everyone for their support.'' A cited reason in their explanation is Google's recent entry into the legal research field.}[INSERT LINKS HERE]
While the systems with the largest corpora are not yet free to the public, this is nevertheless a huge development in the law, as, for the first time in history, it lowers the barriers of legal research such that lay people can easily complete much of the same research as professional legal scholars and attorneys. As a result, the legal world is opened widely to the public, aiding in their understanding of the law, and allowing them to research matters that are of interest to them.

One function that these tools lack, however, is a method for their users to stay up to date with new cases as they are issued by the courts. This leaves researchers with few options if they want or need to stay up to date with an area of the law or with a series of cases. One option that they have is to subscribe to free mailing lists (electronic or otherwise), which aim to keep lawyers up to date with certain areas of the law by sending regular highlights of cases that they feel are relevant.\footnote{NOTE REGARDING INTERVIEW WITH JASON SCHULTZ} This can be a free or inexpensive approach for staying up to date, but the choice of material is not in the hands of users, and separating the wheat from the chaff can be time-consuming, at best. Another option that is available for users is to use existing alert systems, such as Google Alerts, however these can be highly unreliable, as users are subject to the Google's crawl rate, which can be very slow to discover new content, or which can omit relevant information altogether. A final option that is available to supplement or replace the first two, is to simply visit the court websites on a regular basis, and to check there to see if there is any new content that is of interest. For the most part, this is approach works, though it requires a considerable amount of effort, and some courts do not freely publish all of their documents.

In this paper, I introduce a new product, CourtListener.com, which aims to ease this problem by providing a free and open source platform for the aggregation, organization, search and retrieval of legal documents. The aggregation of new court documents is completed by a daemon on a rolling basis, building a huge corpus, and providing the latest cases from the Federal Courts of Appeal within -- on average -- about fifteen minutes from the moment they are published on the court website. From there, the documents are quickly indexed, and RSS feeds and document listings are updated. Finally, at the close of each day and beginning of each week and month, alerts are emailed to registered users informing them about topics that they have identified as relevant. More details about the creation of the corpus, and the design decisions that went into this are available in section \ref{techdecisions}.

In building this system, I spoke with a number of lawyers and academics to understand their needs, and to get input into the design of the system. I will discuss the findings of these informal interviews in section \ref{solutiondesign}, below. Further, after releasing the beta version of the platform, I have received some feedback from users, which I will discuss in section \ref{future}, which is devoted to discussions of the future of the platform.


%However, as the cost of these computerized systems has decreased, few sources have emerged that provide legal research tools for free to the public. 



% the  Perhaps the most revolutionary, however, was the transition to Computer-Aided Legal Research (CALR) that occurred in the early 1980's. 
%While this need originally sprung out of the desire for consistent application of the law, it was not for another five hundred years that 

%there has been a legal Legal documents have been in the American public record for centuries, and there is widespread belief that the legal system functions best when access to these documents is efficient and accurate. 
%Although for the first century of American history there were In 1880, the West Publishing group began to address this need by creating the first volume of the Federal Register. In it, they placed the most recent opinions from the court, and in their introduction, they explain their purpose:
%\begin{quote}
%\end{quote}
%Recent innovations in computer technology make addressing this need possible, and allow legal corpora to be rapidly and accurately searched, however existing solutions fall short of the needs of many users.
%People who wish to stay up to date in certain areas of the law lack tools to do so because the existing solutions are overly complicated, too expensive, and unreliable. 
%In addition, historical court opinions are often difficult to find on court websites. In this paper, I present a new system named CourtListener.com, which automatically visits court websites, downloads and categorizes the latest judgments, and then either sends alerts to users or generates custom Atom feeds. The beta version of this system has been running for more than a month, and is fully functional. It has has also been populated with more than 130,000 judgments, creating a large and growing corpus for users to query.

%Although legal documents have been part of the public record in America for centuries, it is only recently that technology has facilitated the rapid and accurate search and retrieval of those documents. However, although most of these documents are available for free by visiting court websites, because of the decentralized nature of the American court system, people must visit many sites simply to stay up to date on a topic that they find relevant. For those people that are only interested in more obscure areas of the law, this may mean visiting a court website every day or week for months, simply to check if their topic has been mentioned. This means that they must read dozens or hundreds of documents that are of little or no value. Compounding this challenge is the fact that many courts do not make historical documents readily accessible, and so precedent-setting judgments can become hidden from view. Companies have attempted to address these problems by providing alerts to users and massive legal corpora, however current systems are difficult to use and understand and are behind expensive paywalls, which lay people will not pay for and many laywers, academics and non-profits cannot afford. In this paper I present a new system, CourtListener.com, which automatically visits court websites, downloads and categorizes the latest judgments, and then either sends alerts to users or generates custom Atom feeds. The beta version of this system has been running for more than a month, and is fully functional. It has has also been populated with more than 130,000 judgments, creating a large and growing corpus for users to query.





%Although legal documents have been part of the public record in America for centuries, it is only recently that technology has facilitated the rapid and accurate search and retrieval of those documents. However, although most of these documents are available for free by visiting court websites, because of the decentralized nature of the American court system, people must visit many sites simply to stay up to date on a topic that they find relavant. For those people that are only interested in more obscure areas of the law, this may mean visiting a court website every day or week for months, simply to check if their topic has been mentioned. 
%This means that they must read dozens or hundreds of documents that are of little or no value. Compounding this challenge is the fact that many courts do not make historical documents readily accessible, and so precedent-setting judgments can become hidden from view. 
%Companies have attempted to address these problems by providing alerts to users and massive legal corpora, however current systems are difficult to use and understand and are behind expensive paywalls, which lay people will not pay for and many laywers, academics and non-profits cannot afford. In this paper I present a new system, CourtListener.com, which automatically visits court websites, downloads and categor


